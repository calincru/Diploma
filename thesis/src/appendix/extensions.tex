\chapter{iptables match/target extensions}
\label{app:iptables-extensions}

\lstset{
  language=sh,
  basicstyle=\ttfamily,
  keywordstyle=\ttfamily,
  breaklines=true,
  breakatwhitespace=true,
  postbreak=\mbox{\textcolor{red}{$\hookrightarrow$}\space}
}

%%%%%%%%%%%%%%%%%%%
%% MATCH EXTENSIONS
%%%%%%%%%%%%%%%%%%%
\begin{table}[htbp]
  \tiny
  \begin{tabularx}{\textwidth}[t]{ |p{4cm}|X| }
    \hline
    \multicolumn{2}{ |c| }{\normalsize{Match extensions}}
    \\ \hline

    % Comment
    \lstinline{--comment} \emph{comment} &
    Administrator-friendly way of commenting rules.
    \\ \hline

    % Mark
    \lstinline{[!] --mark value[/mask]} &
    Matches the netfilter mark field associated with a packet (which can be set
    using the \emph{MARK} target below).  If a mask is specified, it is ANDed
    with the mark value before the comparison.
    \\ \hline

    % Connmark
    \lstinline{[!] --mark value[/mask]} &
    Matches the netfilter mark field associated with a connection (which can
    then be set using the \emph{CONNMARK} target below).  If a mask is
    specified, it is ANDed with the mark value before the comparison.
    \\ \hline

    % Conntrack
    \lstinline{[!] --ctstate statelist}
    \newline \lstinline{[!] --ctproto l4proto}
    \newline \lstinline{[!] --ctorigsrc address[/mask]}
    \newline \lstinline{[!] --ctorigdst address[/mask]}
    \newline \lstinline{[!] --ctreplsrc address[/mask]}
    \newline \lstinline{[!] --ctrepldst address[/mask]} &
    This module, when combined with connection tracking, allows access to the
    connection tracking state for this packet/connection.  The first option
    matches a comma-separated list of connection states.  The second one
    matches the layer-4 protocol.  The next four match against original/reply
    source/destination address.
    \\ \hline

    % Tcp
    \lstinline{[!] --source-port,--sport port[:port]}
    \newline \lstinline{[!] --destination-port,--dport port[:port]}
    \newline \lstinline{[!] --tcp-flags mask comp}
    \newline \lstinline{[!] --syn}
    \newline \lstinline{[!] --tcp-option number} &
    These extensions can be used if \lstinline{--protocol/-p tcp} is specified,
    besides the usual \lstinline{-m/--match tcp}.
    \newline The first option matches the given source port number or port
    range.
    \newline The second option matches the given destination port number or
    port range.
    \newline The third option matches when the TCP flags are as specified.  The
    first argument, \emph{mask} is the flags which we should examine, given as
    a comma-separated list.  The second argument, \emph{comp}, is a
    comma-separated list of flags which must be set.  Flags are \textbf{SYN ACK
    FIN RST URG PSH ALL NONE}.
    \newline The fourth option matches TCP packets with the SYN bit set and the
    ACK, RST and FIN bits cleared. Such packets are used to request TCP
    connection initiation.
    \newline The fifth option matches if the specified TCP option is set.
    \\ \hline

    % Udp
    \lstinline{[!] --source-port,--sport port[:port]}
    \newline \lstinline{[!] --destination-port,--dport port[:port]} &
    These extensions can be used if \lstinline{--protocol/-p udp} is specified,
    besides the usual \lstinline{-m/--match udp}.
    \newline The first option matches the given source port number or port
    range.
    \newline The second option matches the given destination port number or
    port range.
    \\ \hline

  \end{tabularx}
  \caption{Common iptables match extensions.}
\end{table}

\begin{table}[htbp]
  \tiny
  \begin{tabularx}{\textwidth}[t]{ |X|X| }
    \hline
    \multicolumn{2}{ |c| }{\normalsize{Target extensions}}
    \\ \hline

    % CT
    \lstinline{-j CT --notrack} &
    Allows to set parameters for a packet or its associated connection.  The
    option \lstinline{--notrack} disables connection tracking for this packet.
    \lstinline{-j NOTRACK} is an alias.
    \newline This target is only valid in table \emph{raw}.
    \\ \hline

    % MARK
    \lstinline{-j MARK --set-xmark value[/mask]} &
    It is used to set the Netfilter mark value (32 bits wide) associated with
    the packet.  Option \lstinline{--set-xmark} zeros out the bits given by
    \emph{mask} and XORs \emph{value} into the packet mark (nfmark).
    \newline This target is generally used in table \emph{mangle}, in the
    PREROUTING.
    \\ \hline

    % CONNMARK
    \lstinline{-j CONNMARK --set-xmark value[/mask]}
    \newline \lstinline{-j CONNMARK --save-mark [--nfmask m1] [--ctmask m2]}
    \newline \lstinline{-j CONNMARK --restore-mark [--nfmask m1] [--ctmask m2]} &
    This module sets the netfilter mark value (32 bits wide) associated with a
    connection.  The first option zeros out the bits given by mask and XORs
    value into ctmark.  The second one copies the packet mark (nfmark) to the
    connection mark (ctmark) using the given mask.  The third one is the
    opposite of the second one.
    \\ \hline

    % SNAT
    \lstinline{-j SNAT --to-source ip[-ip][:port[-port]]} &
    Specifies that the source address of the packet should be modified, as well
    as all future packets in this connection.  A single new source IP address
    can be specified, or an inclusive range.  The same applies for the port
    number.  Where possible, no port alteration will occur.
    \newline This target is only valid in table \emph{nat}, in the
    POSTROUTING and INPUT chains.
    \\ \hline

    % DNAT
    \lstinline{-j DNAT --to-destination ip[-ip][:port[-port]]} &
    It is analogous to \emph{SNAT} but for destination IP and port addresses.
    Also, if the port range is not specified, then the destination port will
    never be modified.
    \newline This target is only valid in table \emph{nat}, in the
    PREROUTING and OUTPUT chains.
    \\ \hline

    % MASQUERADE
    \lstinline{-j MASQUERADE --to-ports port[-port]} &
    Masquerading is equivalent to specifying a SNAT mapping to the IP address
    of the interface the packet is going out, but also has the effect that
    connections are forgotten when the interface goes down.
    \newline This target is only valid in table \emph{nat}, in the POSTROUTING
    chain.
    \\ \hline

    % REDIRECT
    \lstinline{-j REDIRECT --to-ports [port[-port]]} &
    It redirects the packet to the machine itself by changing the destination
    IP to the primary address of the incoming interface.  Option
    \lstinline{--to-ports} specifies a destination port or range of ports to
    use.  Without it, the destination port is never altered.
    \newline This target is only valid in table \emph{nat}, in the
    PREROUTING and OUTPUT chains.
    \\ \hline

  \end{tabularx}
  \caption{Common iptables target extensions.}
\end{table}
