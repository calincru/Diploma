\chapter{Evaluation}\label{chapter:eval}

\section{Acceptance tests}



Things to say:
* remind about model-real system equivalence problem
* we cannot achieve that, but we want to ensure that all features that we do
support are implemented and behave as expected. here also mention the black-box
testing approach which we did not do yet, though.

* make sure that its clear what 'acceptance tests' are.
* we performed extensive unit and integration testing which has boosted our
productivity by making it easy to refactor, etc.  The code coverage is
currently 91\% and it is almost as much as it can get.  This means that we do
pass through our tool's code at least one time; however, that should not be
confused with the generated code coverage, and that is why we perform as many
acceptance tests as possible.

* the following is a non-exhaustive and in no particular order list of
scenarios that we want to ensure that work as expected: blabl
* in fact, make this a list of paragraphs following the template: short
description + image.


* the Scala code for a small test is featured in figure ...
\begin{listing}[H]
  \caption{An example of NAT misconfiguration.}
  \label{lst:example}
  \sourcecode{scala}{src/code/simple-nat.scala}
\end{listing}


\section{Performance tests}

Things to say:
* overall results
* framework used, benchmark, etc

* tables/graphs: 1. many rules, on a single filter table; 2. many IPTRouters in
line.
