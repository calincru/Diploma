\chapter{Conclusion}

\paragraph{Summary.} Static verification based on symbolic execution of models
built from network data planes becomes a tractable problem under certain
assumptions.  This is proved by SymNet, the network analysis tool that we used
to verify the SEFL models for iptables-enabled devices generated by our tool,
\TOOL.

Despite the involved semantics of many iptables features, such as user-defined
chains, connection tracking, various variants of network address translation,
we managed to cover the most common ones and show that the resulting models
behave as expected through our acceptance tests suite.  We also show that our
models scale sub-linearly with the number of rules, and that they can be
embedded in large networks that are subject to verification when combined with
a policy-driven approach.

\paragraph{Future work.}
Because our efforts were mostly targeted at having a feature-full
implementation that can take a real world iptables deployment and return a SEFL
model, there is still room for optimization in terms of the generated SEFL
code.  One key observation is that most of the verification time is spent in
the underlying satisfiability solver.  It is the main source of memory
consumption too.  Therefore, fine-tuning our code to generate only the needed
conditions on each execution path (or as close to that as possible) can
dramatically speed up symbolic execution.  However, this is by itself a very
hard problem considering the involved semantics of some iptables rules.

In addition to that, further integration testing needs to be performed.  In
this thesis, we limited our tests mostly to synthetic rules which might miss
certain practices that are common in real deployments.  Nevertheless, this is
just a small part of a larger project that aims to build a provably correct
model of an OpenStack deployment and have it verified with SymNet.  Therefore,
integrating it with the other networking components used as part of Neutron
(e.g. Open vSwitch) will be the immediate follow-up challenge.
