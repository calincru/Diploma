\chapter{Introduction}


\section{Motivation}

Modern networks have grown to the point where understanding the way their
components interact with each other is beyond most administrators' ability. In
fact, studies have shown that more than 62\% of network failures today are due
to network misconfiguration~\cite{alimi2008shadow}.  In addition to that,
hardware-based network appliances, while essential in today's operational
networks, bring about even more complexity: they are expensive, complex to
manage and introduce new failure modes~\cite{sherry2012making}.

At the same time, more and more x86 infrastructure is being deployed for
networking purposes. One particular direction which has gained popularity lately
is Network Function Virtualization (NFV)\abbrev{NFV}{Network Function
Virtualization}, a network architecture concept that unifies virtualization and
networking by running the functionality provided by middleboxes as software on
commodity hardware~\cite{martins2014clickos, stoenescu2015net}. The idea has
been taken even further by operators who have spotted the potential cost
benefits and are planning to extend their services to become cloud providers
specialized for in-network processing~\cite{stoenescu2015net}, which is bound
to intensify this trend.

When speaking about running network functions on x86 hardware, we immediately
have to refer to \textbf{netfilter}, the customizable framework provided by the
Linux kernel that offers various functions and operations for packet filtering,
packet mangling and Network Address Translation (NAT)\abbrev{NAT}{Network
Address Translation}, thus enabling most of the functions featured by dedicated
firewalls. For IPv4 traffic, these functions are administrated through
\textbf{iptables}, its better-known user-space counterpart, allowing us to
refer to them almost interchangeably.

To understand the increased interest in iptables we have to look once again at
the cloud computing world.  OpenStack, an open-source software platform for
cloud computing, relies heavily for its networking component, Neutron, on
iptables which is used for filtering, NAT and implementing security groups, to
name a few, both in its core infrastructure and, on top of that, by its
users~\cite{denton2014learning}.  One common problem that arises here is that
tool-generated rules do not always compose as expected with the ones managed by
users.

In this context, it is clear that applying formal verification methods to
networking problems is motivated not just by the theoretical challenges it
poses, but also by practical needs.  Driven by this insight, we propose a way
to model an iptables-enabled device (a software router built around netfilter)
using SEFL\abbrev{SEFL}{Symbolic Execution Friendly Language}, a network
description language that is \textbf{symbolic execution} friendly, and SymNet,
a tool for static network checking~\cite{stoenescu2016symnet,
stoenescu2013symnet}.  Together, they assemble into a state-of-the-art
framework that enables scalable verification for stateful networks.

The resulting tool, naturally called \TOOL, compiles iptables deployments to
SEFL models, and sums up to approximately 8k lines of Scala code (including
unit tests).  These models, in turn, can then be plugged into large networks
and reveal configuration errors or, more generally put, policy inconsistencies.
Therefore, it is a step forward towards the goal of verifying networks which
are increasingly complex.


\section{Goals and non-goals}

The end goal of this project is to be able to verify (using SEFL and SymNet)
certain properties of networks that integrate iptables as part of their packet
processing elements.  In the two lists below we split it into smaller, more
detailed steps that we aim at, on one hand, and that we want to avoid or simply
are not interested in, on the other hand.

\paragraph{Goals}
\begin{itemize}
  \item item 1
  \item item 2
\end{itemize}

\paragraph{Non-goals}
\begin{itemize}
  \item item 1
  \item item 2
\end{itemize}
