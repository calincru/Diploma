\chapter{Design and implementation}\label{chapter:design}

\todo{chapter overview; will present the high-level model and increase amount
of details gradually -- also show a pseudocode of the most important one, the
traversal of a chain of rules; then, implementation details follow, compiler-like
design, etc; we then go into more details about a few extensions and finally,
limitations of the model are discussed}

\section{Towards a model}
\todo{Give some overview of where we are currently standing and how we plan to
proceed. then.. proceed; start from router, augment it; make clear what virtual
devices are (we used this term before blabla); reiterate the encapsulation
idea}
\todo{once we show the first prototype, take some time to discuss the design:
parsing, validation, code generation, their importance, how they play together,
the fact that the table/chain/rule parsing as well as the "target" (the links
as defined in that figure, etc) are always the same, but we need to consider
the extensions-based design, etc. Mention that more details about the actual
implementation are given in section implementation}.
\todo{finish by saying that our model in figure XX hides some quite involved
logic caused by the support for user-specified chains and others, which are
discussed next..}

\subsection{User-defined chains}
\todo{explain WHY they complicate the simple model in the previous section.
explain that our final model reflects their exact semantics}
\todo{what is an Iptables Virtual Device (IVD); what is a Chain IVD; cover in
detail all components of a Chain IVD: input dispatcher, contiguous IVDs, output
dispatchers}

\subsection{Network Address Translation}
\todo{start with all kinds of NAT supported? SNAT, DNAT, MASQUERADE (mention
that it is essentially SNAT, since we model data plane only), REDIRECT}
\todo{WHY we treat NAT separately? mention the fact that for each flow, NAT
tables are consulted once only, and then the applied rule is automatically
re-applied; this means that custom logic that SKIPS the table has to be added;
show how this changes the model we reached in the previous section; add the
Chain IVD Initializer components.}
\todo{besides that, reply-packets have to be reverse-NAT'ed}

\subsection{Connection tracking}
\todo{show how we introduce a new \emph{virtual device} that implements the
connection tracking logic. mention that it is currently limited, only NEW to
ESTABLISHED is handled; explain how it works, more precisely; also mention that
a great limitation is discussed in
\labelindexref{Section}{sub-sec:related-state}.}

\bigskip
\todo{conclude by saying that it finalizes the current state of our model,
which is quite involved compared to the initial one we devised at the beginning
of this section.}


\section{Implementation}\label{sec:implementation}
\todo{Could also mention the size of the project; 4k + 4k, "out of tree"
(SymNet); not really "test-driven", but more like "tested"}
\todo{this chapter shows how some of the things discussed in the previous
section are implemented;}
\todo{add implementation-detailed diagrams of the 'iptelement' hierarchy as
well as of the 'virtualdevice' hierarchy; mention composite approach, etc.}
\todo{(inspired by todo 1 above) Discuss what the interface with SymNet is. --
these parts (Instruction/:==:, etc) will be added to a separate "api" package,
to be easily accessed by other models and for better code organization.}
\todo{Before delving into the next subsections, could also mention code
structure, 3 main directories (core, extensions - one directory for each
extension -, virtdev), the driver in the root}

\subsection{Parsing}
\todo{talk about parsing, table parsing, chain parsing, ParsingContext, rule
parsing, match/target parsing, their extensions, how they are extended, etc}
\todo{for rule parsing, mention the functionality behind the 'match extension
activation', flag -m or --match}
\todo{Monadic, Parsec, functional, haskell etc}

\subsection{Validation}
\todo{WHY it is needed; say that it is analogous to semantic analysis in
compilers}
\todo{use cases: unordered chains (needed), port range validity, certain chains
in certain tables, certain rules in certain tables/chains, etc}
\todo{how it works, validate() function, validateIf variant, ValidationContext}

\subsection{Code generation}
\todo{SeflGenOptions trait, the variant for match extensions and the one for
target extensions}


\section{Extensions}
\todo{is this section still needed? maybe add some examples in the previous
sections, rather than adding another one}
\subsection{TCP and UDP}
\subsection{MARK and CONNMARK}


\section{Limitations}
\todo{We build precise models, as far as our toolset (SymNet, SEFL) permits.
However, there are certain limitations which are discussed next.}

\subsection{The local process}
\todo{say that we are not interested in modelling the local process, as a
correct, or at least close to one, model of that would be the kernel itself,
plus application specific logic.}
\todo{however, one trick we do to allow 'simulation' of traffic generated by
some applications is to expose its output port from figure XX. this permits
starting symbolic execution by injecting a symbolic packet on that port and see
what happens, etc.}

\subsection{The RELATED connection state}\label{sub-sec:related-state}
\todo{say again what its purpose should be; give an use-case}
\todo{Inherent limitation caused by the independence between different flows.}
