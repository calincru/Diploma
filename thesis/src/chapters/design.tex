\chapter{Design and implementation}
In this chapter we delve into the design and the implementation details of
\textbf{\texttt{iptables-to-sefl}}, the tool we built to generate SEFL models
from iptables configurations.

\todo{by-section overview}


\section{Design overview}


\section{Implementation}
\todo{Could also mention the size of the project; 4k + 4k, "out of tree"
(SymNet); not really "test-driven", but more like "tested"}
\todo{this chapter shows how some of the things discussed in the previous
section are implemented;}
\todo{add implementation-detailed diagrams of the 'iptelement' hierarchy as
well as of the 'virtualdevice' hierarchy; mention composite approach, etc.}
\todo{(inspired by todo 1 above) Discuss what the interface with SymNet is. --
these parts (Instruction/:==:, etc) will be added to a separate "api" package,
to be easily accessed by other models and for better code organization.}
\todo{Before delving into the next subsections, could also mention code
structure, 3 main directories (core, extensions - one directory for each
extension -, virtdev), the driver in the root}


\subsection{Parsing}
\todo{talk about parsing, table parsing, chain parsing, ParsingContext, rule
parsing, match/target parsing, their extensions, how they are extended, etc}
\todo{for rule parsing, mention the functionality behind the 'match extension
activation', flag -m or --match}
\todo{Monadic, Parsec, functional, haskell etc}


\subsection{Validation}
\todo{WHY it is needed; say that it is analogous to semantic analysis in
compilers}
\todo{use cases: unordered chains (needed), port range validity, certain chains
in certain tables, certain rules in certain tables/chains, etc}
\todo{how it works, validate() function, validateIf variant, ValidationContext}


\subsection{Code generation}
\todo{SeflGenOptions trait, the variant for match extensions and the one for
target extensions}


\section{Extensions}
\todo{is this section still needed? maybe add some examples in the previous
sections, rather than adding another one}
\subsection{TCP and UDP}
\subsection{MARK and CONNMARK}
