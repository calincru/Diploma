\documentclass[twoside, 11pt, a4paper]{article}

%%%%%%%%%%%%%%%%%%%%%%%%%%%%%%%%%%%%%%%%%%%%%%%%%%%%%%%%%%%%%%%%%%
% Any additional packages needed should be included after csmr.  %
% Note that csmr.sty includes epsfig, amssymb and graphicx,      %
% and defines many common macros, such as 'proof' and 'example'. %
%%%%%%%%%%%%%%%%%%%%%%%%%%%%%%%%%%%%%%%%%%%%%%%%%%%%%%%%%%%%%%%%%%

\usepackage{csmr}
\usepackage[cp1250]{inputenc}
\usepackage{fancyhdr}
\usepackage[super]{nth}
\usepackage{paralist}
\usepackage{wrapfig}
\usepackage{listings}

% Definitions of handy macros can go here

%\newcommand{\dataset}{{\cal D}}
%\newcommand{\fracpartial}[2]{\frac{\partial #1}{\partial  #2}}

% Heading arguments are {volume}{nmber}{submitted}{published}{author-full-names}

% \csmrheading{1}{1}{2017}

% Short headings should be running head and authors last names

\ShortHeadings{1}{1}{2017}{Formal Analysis of iptables Configurations}{Calin Cruceru}

\begin{document}

\title{Formal Analysis of iptables Configurations}

\author{\name Calin Cruceru\\
       \addr University POLITEHNICA of Bucharest\\
       Faculty of Automatic Control and Computers, Computer Science Department\\
       \email E-mail: calin.cruceru@stud.acs.upb.ro}

\maketitle

\begin{abstract}

As modern networks become increasingly complex and more and more operators move
towards Network Function Virtualization as a way of reducing costs for
purchasing and upgrading middleboxes, verifying certain properties of
networks built around \texttt{iptables}-enabled devices becomes one of the
top priorities in network verification today.  Therefore, in this article we
create a model of an iptables-enhanced router which can be fed to SymNet, a
fast symbolic execution tool for stateful networks and show how it can be
used in practice.

\end{abstract}

\begin{keywords}
network verification, symbolic execution, iptables, Scala
\end{keywords}

\section{Introduction}
Modern networks have grown to the point where understanding the way their
components interact with each other is beyond most administrators' ability. In
fact, studies have shown that more than 62\% of network failures today are due
to network misconfiguration \cite{alimi2008shadow}. Consequently, applying
formal verification methods to networking problems has not been motivated just
by the theoretical challenges it poses, but also by a practical need.

In addition to that, more and more x86 infrastructure is being deployed for
networking purposes. A specific direction which has gained popularity lately
is Network Function Virtualization (NFV), a network architecture concept that
unifies virtualization and networking by running the functionality provided by
middleboxes as software on commodity hardware \cite{stoenescu2015net}. The idea
has been taken even further by operators who have spotted the potential cost
benefits and are planning to extend their services to become cloud providers
specialized for in-network processing \cite{stoenescu2015net}.

When speaking about running network functions on x86 hardware, we immediately
have to refer to \texttt{Netfilter}, the customizable framework provided by the
Linux kernel that offers various functions and operations for packet filtering,
packet mangling and Network Address Translation (NAT), thus enabling most of
the functions featured by dedicated firewalls. These functions are
administrated through \texttt{iptables}, its better-known user-space
counterpart, allowing us to refer to them almost interchangeably.

To understand the increased interest in iptables we have to look once again at
the cloud computing world.  OpenStack, an open-source software platform for
cloud computing, relies heavily for its networking component, Neutron, on
iptables which is used for filtering and NAT both in its core infrastructure
and, on top of that, by its users. One common problem is that tool-generated
rules do not always compose as expected with the ones managed by users.

We propose a way to model an iptables-enabled device (a software router built
around Netfilter) using SEFL \cite{stoenescu2016symnet}, a network description
language that is \emph{symbolic execution} friendly and SymNet, a tool for
static network checking. It can then be easily plugged in large networks and
reveal the aforementioned conflicts.  We have implemented this model in Scala
and early tests show that it can catch common iptables misconfigurations.

\section{Network Verification}
Network correctness can refer to a couple of different things. The most
well-studied problem is \emph{reachability}, which can be stated as simple as

\begin{quote}
\emph{Given two points in the network, A and B, can A reach B?}
\end{quote}

It might be the case that we want to ensure that this \emph{does} happen, or
that it \emph{does not}.  It can also be adjusted to more restricted scenarios,
such as specifying the \nth{4} layer protocol to use.  Another, rather general
question that we might ask is

\begin{quote}
\emph{Is the policy of the operator properly enforced?}
\end{quote}

It is abstract because \emph{policy} is not well-defined in this context.  It
could refer to some access control policy, such as \emph{no department should
have access to the management VLAN}, or a security policy as in \emph{all
packets entering a tunnel end-point should be encrypted}. In fact, there is
ongoing effort to bring policy specification closer to the natural language and
integrate it with existing network verification systems.

To answer these questions, we build a \emph{model} of the network.  We start
with a constraint: we only consider the \emph{data plane} of the network
elements.  That is, we verify network's functionality at an instant, as opposed
to over-time, in which case the entire control plane (i.e. dynamic routing
algorithms, SDN controllers, etc) should be modeled, which is currently not
feasible. This is called \textbf{static checking} \cite{stoenescu2013symnet}
and it is the main trade-off made to enable tractable network verification.

Implementation-wise, \textbf{symbolic execution} offers the best guarantees,
considering its long and successful history in \emph{software verification}
\cite{cadar2008klee} and the fact that careful crafting of network description
languages can get around its worst-case exponential complexity. Symbolic
Execution Friendly Language (SEFL) \cite{stoenescu2016symnet} is such a
language. Its truly innovative insight is that focusing on \emph{flows} instead
of packets increases its expressiveness by extending the set of networks that
it can model to include stateful ones. Therefore, network functions are
described as simple \emph{flow transformations}.

SymNet is the network analysis tool that takes as input a network model given
as the SEFL descriptions of all network elements and their connections. It
initiates symbolic execution by injecting an empty (symbolic) packet on a
user-specified port, symbolically executes SEFL instructions associated with it
and then follows the outgoing links.  It does so until either
\begin{inparaenum}[i)]
  \item there are no outgoing links,
  \item an explicit SEFL \emph{Fail} instruction is encountered, or
  \item a loop is detected.
\end{inparaenum}
At the end, it outputs all possible symbolic paths, as well as their
constraints, port history and metadata (rewritings performed by NATs, tunnels,
encryption, etc).

Results show that SymNet, powered by SEFL models, scales linearly with the size
of the modeled network \cite{stoenescu2016symnet} and, thus, can be used to
verify large networks.  The requirement is to be able to build models for all
the network elements they consist of.

\section{iptables}
iptables is an administration tool for IP packet filtering/NAT based on
Netfilter, a collection of modules and customizable hooks that powers
Linux-based firewalls. While iptables is just the user-space application that
forwards the requests to Netfilter, where all rules and the logic for
filtering, mangling and NAT reside, it is common to implicitly include all of
its underpinnings when referring to \emph{iptables}.

\begin{wrapfigure}{r}{0.5\textwidth}
  \vspace{-20pt}
  \begin{center}
    \includegraphics[width=0.48\textwidth]{tables_traverse}
  \end{center}
  \vspace{-20pt}
  \caption{iptables forwarding path}
  \vspace{-10pt}
  \label{iptables}
\end{wrapfigure}

iptables has a modular organization (Figure \ref{iptables}), making it easy to
control \textbf{which} traffic is matched, \textbf{when} it can be matched
against and \textbf{what} happens with it. Its building blocks are the
following:

\begin{itemize}
  \item \emph{Rule}, composed of matching conditions and a target that gives
    the action performed when all of them are true.

  \item \emph{Chain}, an ordered list of rules which are checked against
    sequentially.  If it is one of the 5 built-in chains, it has a well-defined
    place during packet processing when it is traversed.  For example, NAT can
    be performed only in the \texttt{PREROUTING}, \texttt{OUTPUT} and
    \texttt{POSTROUTING} chains. New chains can be defined and used as targets
    in rules.

  \item \emph{Table}, contains multiple chains and restricts what is allowed to
    do in a rule. For example, in the \texttt{filter} table we can only
    \texttt{DROP} and \texttt{ACCEPT} packets, while in the \texttt{nat} table
    we can do Source NAT (\texttt{SNAT}) and Destination NAT (\texttt{DNAT}).
    Another constraint is that it is not possible to jump to a user-defined
    chain from another table.
\end{itemize}

We only model a strict subset of iptables. This is because, firstly, some
features depend on the local process, such as the special target \texttt{QUEUE}
which passes the packet to userspace.  Secondly, some other features depend
on the control plane, which is ignored, as explained before; an example is the
\texttt{MASQUERADE} target in the \texttt{nat} table which is similar to
\texttt{SNAT} with the difference that it is used for dynamically assigned IP
connections.  Since we are looking only at a snapshot of an iptables
deployment, the way the IP addresses are obtained is not important.

The high-level model of an iptables-enabled router (Figure
\ref{iptables-model}) looks similar to the previous flowchart, only that each
circle that represents a chain is now a SEFL \emph{element} that contains the
entire logic for traversing a chain.  There are also three routing decisions,
as opposed to only one in a regular router.  The way a router is modeled using
SEFL (and, implicitly, its routing decision \emph{element}), has been already
discussed in \cite{stoenescu2016symnet}, the only difference is that we add
special logic for checking if the destination IP address of the analyzed packet
is a local one, which results in forwarding the packet to the local process.

\begin{figure}[t]
  \centering
  \includegraphics[width=0.8\textwidth]{iptables-model}
  \caption{iptables model}
  \label{iptables-model}
\end{figure}

As mentioned before, the local process is neither modeled, nor verified.
However, the chains corresponding to its output are still of interest as they
can be tested by exporting the input port of the \emph{Local RD} element and
injecting symbolic packets into it.

At a first thought, it would seem that modeling chain traversal is as simple as
a big If/Then/Else statement which has a "depth" equal to the number of rules.
However, besides the exponential number of constraints it generates, there
is another, more subtle problem: besides the possibility to \emph{jump} to a
user-defined chain, it is also possible to \texttt{RETURN} back from it and
continue traversing the original chain from the successor of the rule that
caused the jump. This means that we have to \emph{remember} that specific rule
as part of the flow and add additional links between the chains which are
\emph{connected} in this way.

\section{Implementation}
The tool that resulted from the above modelling is called
\texttt{iptables-to-sefl} \cite{github-repo} and is essentially a translator
that targets SymNet by generating a SEFL model of an iptables-enabled device.
Its design resembles a compiler; it takes as input raw tables which contain
multiple chains, and each chain can have multiple rules.  It then goes through
the following pipeline:

\begin{itemize}
  \item \emph{Parsing}.  Given that rules are simple enough, the parser is a
    monadic LL(*) one, inspired by Parser \cite{leijen2002parsec}.  The result
    is an in-memory model of the given rules.
  \item \emph{Validation}.  This step is analogue to semantic analysis in
    compilers; it ensures that the given rules are correctly formatted.  Things
    which are hard to catch during parsing require a separate validation step
    include:
    \begin{itemize}
      \item Forward references to user-defined chains
      \item Catching missing built-in chains and tables (\texttt{filter} and
        \texttt{nat} must be provided)
      \item Protocols mentioned as part of the \texttt{-p <protocol>} match
        extension must be supported
    \end{itemize}
  \item \emph{SEFL code generation}.  The code for each rule is kept separately
    from the actual model (the links between chains which jump to one another,
    etc), making it straightforward to add new match/target extensions.  The
    \emph{codegen} part refers precisely to that: generating SEFL code for each
    supported match and target.  For instance, the code which matches the
    source address of a packet (e.g. \texttt{-s 192.168.18.0/24} in iptables)
    is shown below.
\end{itemize}

\begin{lstlisting}[language=scala,label={codegen}]
override def seflConstrain(ip: Ipv4): Option[Instruction] = {
  val (lower, upper) = ip.toHostRange

  Some(Constrain(IPSrc, :&:(:>=:(ConstantValue(lower.host)),
                            :<=:(ConstantValue(upper.host)))))
}
\end{lstlisting}

Finally, the resulted box which hides all its functionality internally can be
linked to other network elements through its exposed input and output ports,
leading to more complex networks ready to be fed to SymNet.

\section{Future Work}
One feature that could enable many other extensions and is not yet supported is
\emph{connection tracking}.  This is a feature that would make great use of
SEFL's expressiveness as it is a stateful processing.  Extending
iptables-to-sefl to support it is one of the top priorities.

It is followed in priority order by optimizing SEFL code which is generated for
chain traversal; currently, as aforementioned, it is simply a big If/Then/Else
statement, which generates too many constraints.  However, a more clever
algorithm could probably be devised in order to find and remove redundant
constraints.  In fact, work in this direction is already being conducted, as
similar code is generated by different scenarios.

Lastly, \emph{additional testing} is required to make sure that the model
scales to very large iptables deployments and for various use cases.

\vskip 0.2in
\bibliographystyle{plain}
\bibliography{bibliography}

\end{document}
